\documentclass[10pt,aspectratio=169,xcolor=x11names,compress,dvipsnames]{beamer}

\definecolor{myblue}{rgb}{0.19,0.25,0.41}

\usetheme[progressbar=frametitle]{metropolis}

\useoutertheme{metropolis}
\useinnertheme{metropolis}
\setbeamercolor{background canvas}{bg=white}

%\usecolortheme[named=myblue]{structure}
\setbeamercolor{frametitle}{bg=myblue}

\definecolor{myred}{rgb}{0.82, 0.1, 0.26}
\setbeamercolor{progress bar}{fg=myred}

 \usepackage[dvipsnames]{xcolor}
% for graphs
\usepackage[demo]{graphicx}
\usepackage{subfig}
\usepackage{comment} 

%Package for tables
\usepackage[section]{placeins}
\usepackage{float}
\usepackage{multirow}
%\newcommand{\textbf}[1]{\textnormal{\textbf{#1}}}
\usepackage{booktabs} % Allows the use of \toprule, \midrule and \bottomrule in tables
\usepackage[para,online,flushleft]{threeparttable}


\usepackage{hyperref}
\hypersetup{
    colorlinks=true,
    linkcolor=blue,
    filecolor=magenta,      
    urlcolor=cyan,
}
 
\usepackage[cache=false]{minted}

\usepackage{amsthm}
\usepackage{amssymb}

\newtheorem*{exercise}{Exercise}

\renewcommand{\qedsymbol}{$\blacksquare$}



\csname toclevel@#2\endcsname
\title[]{Lecture 2: Fundamentals of Programming \\ Programming for Economics---ECNM10106}
\author{Albert Rodriguez-Sala \& Jacob  Adenbaum \\
School of Economics, The University of Edinburgh }
\date{Winter 2023}

\begin{document}



%-----------------------------------------------------------------Qa

    \maketitle


\begin{frame}{This class}
 \tableofcontents 
\end{frame}  


\begin{frame}{Basic synax in Python}

\begin{itemize}
    \item each line in a Python script is a\textbf{ statement}. Unless broken with $\setminus$ or in lists.
    \item Naming convention in Python. \textbf{Identifier} is the name given to our programming elements (variables, functions, classes, modules, packages, etc.). 
    \begin{itemize}
        \item An \textbf{identifier should start} with either an alphabet letter or an underscore (\_). 
        \item After that, more than one alphabet letter (a-z or A-Z), digits (0-9), or underscores may be used to form an identifier. No other characters are allowed.
        \item Identifiers in Python are \textbf{case sensitive}: variables $list_1$ and $List_1$ are different.
        
        
    \end{itemize}
    \item \textbf{Display Output}: the print() function serves as an output statement in Python.
\end{itemize}




    
\end{frame}



\begin{frame}{Indexing and slicing in Python}
    objects in Python can be non-sliceable or sliceable. \textbf{Sliceable objects}---as strings, lists, tuples,arrays, dataframes, etc--- are objects containing \textit{indexed elements}---as cells in an array, elements in a list, columns/rows in a dataframe.

   \textbf{ Caution: indexing in Python starts with zero!!!}
   \begin{itemize}
       \item 	First element [0], second element[1], nth element [n-1].
       \item In a "2-D" object: [0,:] first set of elements, like the first row in an array or a dataframe. [0,0] first element---first row, first column. [:,0] first column.
       
       \item To recover last elements we use the minus: last elment[-1], previous last element [-2], [:,-1] entire last column.
       \item \textbf{Slicing}: to recover closed intervals $[i,i+n]$ on $x$ we use the syntax $x[i-1:i+n]$.	From the 2th to the 4th element: [1:4]. From the 4th element to the last one [3:]. Last 2 rows, first three columns [-2:, 0:3]
   \end{itemize}
\end{frame}


\begin{frame}{Python works with identation}

\begin{itemize}
    \item Python uses \textbf{whitespace }and \textbf{indentation} to construct the \textbf{code structure}. Thus, identation is the equivalent to the curly brackets \{\} in Stata, R, and other programs.
    \item Identation helps preserve the order and readiness of the code. Identation also makes it faster to write codes and avoid missing \} or \{.
    \item \textbf{Sypder has automated identation}. You should use the tab whenever you want to indent (each tab one indent).
\end{itemize}
    
    


 
\end{frame}

 \begin{frame}{Blocks of code}
Identation rules to indicate blocks of code.
\begin{itemize}
    \item Use the colon : to start a block and press Enter.
    \item All the lines in a block must use the \textbf{same indentation}, either 4 space or a tab.
    \item A block can have \textbf{inner blocks} with next-level indentation.
\end{itemize}



\end{frame}


\begin{frame}{Frame Title}
    
\begin{example}
\text{Compute the mean of a matrix  $ A=  \begin{bmatrix} 1 & 0 & 1 \\ 0 & 0 & 1 \end{bmatrix} $}
\begin{minted}{python}
# Import NumPy library to work with matrices/arrays.
import numpy as np 
A = np.array([[0,1,0],[0,1,1]])  #create the 2-D array---i.e. matrix
# Procedural approach. We call np.mean function to act on A data.
 np.mean(A)  
 0.5
# OOP approach. We call the array-method mean to act on the array-object A.
 A.mean()  
 0.5
\end{minted}


\end{example}

\end{frame}



\end{document}